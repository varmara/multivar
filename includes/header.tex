%%%%%%%%%%%%%%%%%%%%%%%%%%%%%%%%%%%%%%%%%%%%%%%%%%%%%%%%%%%%%%%%%%%%%%%%%%%%%%%%%%%%%%%
%     <-- выше автоматическая преамбула  |  пользовательская преамбула ниже --->      %
%%%%%%%%%%%%%%%%%%%%%%%%%%%%%%%%%%%%%%%%%%%%%%%%%%%%%%%%%%%%%%%%%%%%%%%%%%%%%%%%%%%%%%%

%%% Работа с русским языком
\usepackage{cmap}					    % поиск в PDF
\usepackage{mathtext} 				% русские буквы в формулах

% Для разметки слайдов и картинок
\usepackage{tikz}
\usepackage{animate}

% Для двойных прямых кавычек в Verbatim (не работает почему-то)
\usepackage{upquote}

% % Работа с прочими шрифтами
% \usepackage{fontspec}      %% УЖЕ ЗАГРУЖЕН В ПРЕАМБУЛЕ подготавливает загрузку шрифтов Open Type, True Type и др.
\setmainfont{DejaVu Sans} %% задаёт основной шрифт документа
\setsansfont{DejaVu Sans} %% задаёт шрифт без засечек
\setmonofont[Scale=0.7]{DejaVu Sans Mono} %% моноширинный шрифт\

% Межстрочный интервал в verabtim
\makeatletter
\def\verbatim@font{\linespread{0.7}\normalfont\ttfamily}
\makeatother

% Межабзацный интервал в columns
\makeatletter
\newcommand{\@minipagerestore}{\setlength{\parskip}{4pt plus 2pt minus 1pt}}
\makeatother

% \usepackage{hyperref} %% УЖЕ ЗАГРУЖЕН В ПРЕАМБУЛЕ
\definecolor{links}{HTML}{2A1B81}
\hypersetup{colorlinks=true,linkcolor=,urlcolor=links,pdfview=FitH,pdfpagelayout=SinglePage, unicode=true,breaklinks=true}

% Тильда в тексте
\newcommand{\mytilde}{\char`~\:}
% argmin и argmax
\DeclareMathOperator*{\argmax}{argmax}
\DeclareMathOperator*{\argmin}{argmin}

% Более простой ввод команд для двухколоночной верстки
\newcommand{\columnsbegin}{\vspace{-0.5\baselineskip}\begin{columns}[t,onlytextwidth]}
\newcommand{\columnsend}{\end{columns}}
\newcommand{\columnbegin}{\begin{column}}
\newcommand{\columnend}{\end{column}}
\newcommand{\blockbegin}{\begin{block}}
\newcommand{\blockend}{\end{block}}

% allows to add alignment keys to \includegraphics
\usepackage[export]{adjustbox}

% Работа с таблицами %%%%%%%%%%%%%%

\usepackage{adjustbox}

% Новые типы колонок для окружения tabular, чтобы можно было задавать их ширину
% Например, \begin{tabular}{ L{2.3cm} C{2cm} C{1.5cm} C{2.5cm} C{4cm}}
\usepackage{array}
\renewcommand{\arraystretch}{1.5}
\newcolumntype{L}[1]{>{\raggedright\let\newline\\\arraybackslash\hspace{0pt}}m{#1}}
\newcolumntype{C}[1]{>{\centering\let\newline\\\arraybackslash\hspace{0pt}}m{#1}}
\newcolumntype{R}[1]{>{\raggedleft\let\newline\\\arraybackslash\hspace{0pt}}m{#1}}

% Для таблиц в tabularx окружении при помощи пакета huxtable
\usepackage{tabularx}

% Прочие настройки beamer %%%%%%%%%%%%

% decrease text margins
\setbeamersize{text margin left = 8pt, text margin right = 16pt}

% Галочка в тексте
\def\checkmark{\tikz\fill[scale=0.4](0,.35) -- (.25,0) -- (1,.7) -- (.25,.15) -- cycle;}

% Цвета
% Посмотреть и конвертировать можно здесь https://convertingcolors.com/
%                       RGB                       HEX    nearest R colour
\definecolor{myblack}{RGB}{30,30,30}           % 1e1e1e "gray12"
\definecolor{mygrey}{RGB}{191,191,191}         % BFBFBF "gray75"
\definecolor{myteal}{RGB}{188, 216, 221}       % BCD8DD "powderblue"
\definecolor{mylightteal}{RGB}{222,236,238}    % DEECEE "azure2"
\definecolor{myred}{RGB}{203,55,87}            % CB3757 "maroon"
\definecolor{mygreen}{RGB}{168,207,166}        % A8CFA6 "darkseagreen3"
\definecolor{mydarkgreen}{RGB}{96,157,131}     % 609D83 "paleturquoise4"
\definecolor{myblue}{RGB}{98,122,235}          % 627AEB "cornflowerblue"
\definecolor{mydarkblue}{RGB}{24,112,184}      % 1870B8 "dodgerblue3"
\definecolor{myviolet}{RGB}{212,207,232}       % D4CFE8
\definecolor{mydarkviolet}{RGB}{101,118,185}   % 6576B9
\definecolor{myyellow}{RGB}{254,192,15}        % FEC00F "darkgoldenrod1"
\definecolor{mysteelblue}{RGB}{70,130,180}     % 4682B4 "steelblue"

% Для подсветки элементов формул
\usepackage[beamer,customcolors,norndcorners]{hf-tikz}
\hfsetfillcolor{myviolet}
\hfsetbordercolor{white}

% \logo{\includegraphics[height=0.3cm]{assets/Linmod_logo.png}}

% Переопределяем бимеровский footline
\setbeamertemplate{footline}{
  \quad\hfill\insertframenumber/\inserttotalframenumber\strut\quad
}
% Заголовок на титульном слайде
\setbeamercolor{titlelike}{parent=palette primary,fg=myblack,bg=}

% Блоки
\setbeamercolor{block title}{fg=myblack, bg=}

% Заголовки слайдов
\setbeamercolor{frametitle}{bg=}
\setbeamertemplate{frametitle}{
  \vspace{1cm}
  \color{myblack}\bfseries\insertframetitle%
}
% Маркеры элементов списка
\setbeamertemplate{itemize items}{\color{myred}\normalsize$\bullet$}


%%%%%%%%%%%%%%%%%%%%%%%%%%%%%%%%%%%%%%%%%%%%%%%%%%%%%%%%%%%%%%%%%%%%%%%%%%%%

